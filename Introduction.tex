\chapter*{General introduction}
 The rapid technological development has brought an excessive amount of data throughout entertainment sectors, fashion industries and social media platforms. People are facing a growing problem of information overload because of the excessive amount of data they receive. With endless choices at their fingertips, individuals find it increasingly difficult to sift through options and make informed decisions. In this context, recommendation systems and AI assistants have emerged as essential tools in filtering massive data and delivering personalized suggestions. With a growing societal focus on health and wellness, these technologies have become increasingly relevant for the nutrition domain. By providing scientifically grounded guidance, it address conflicting dietary information and supports healthier lifestyle choices.\\


\par From this perspective, The company ONRTECH is currently working on a major project called Vitamiurse, which aims to provide a mobile application capable of delivering personalized recommendations based on users' dietary preferences.

Within this framework, our contribution focused on developing the components related to the recommendation system and the intelligent assistant.

This report presents a comprehensive documentation of the work carried out
during my internship. It consists of four separate chapters, each addressing distinct aspects of my project. The first chapter introduces the general context of the project, presents the host company and outlines the objectives of the internship. It also describes the methodology adopted and the project architecture.\\
The second chapter focuses on the design and implementation of the data pipeline using an advanced ETL framework. The third chapter describes the development of a hybrid recommendation system. This hybrid system combines collaborative
filtering, content-based approaches, and real-time analytics. The last chapter presents the design and integration of the intelligent conversational assistant. This chapter highlights its core functionalities, its interaction with users, and its role in providing personalized nutritional guidance.

In conclusion, we mention the various strengths of this project and summarize all the work accomplished while proposing some perspectives that could further improve our solution.

