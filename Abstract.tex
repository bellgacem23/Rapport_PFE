\chapter*{Abstract}

This graduation project, carried out at ONRTECH between February and August 2025, is part of our final year of the Professional Master’s degree in Data Science at the Higher Institute of Computer Science and Mathematics of Monastir (ISIMM). It focuses on the development of \textit{VitamiNurse}, an innovative application designed to help users make informed and personalized food choices. Based on artificial intelligence, it evaluates products, suggests healthier alternatives, and delivers real-time interactive support.

Developed using Scrumban and CRISP-DM methodologies, the VitamiNurse project is structured around several components, including a data pipeline, a hybrid recommendation system, and an intelligent chatbot. Together, these elements create a dynamic and user-friendly tool enabling users to make healthier and more balanced food choices.

\noindent\textbf{Keywords:} ETL, Recommendation Systems (RS), Machine Learning, Generative AI, Chatbot, ANN, SentenceTransformers.

\newpage
\chapter*{Résumé}
Ce projet de fin d’études, réalisé à l’ONRTECH entre février et août 2025, s’inscrit dans le cadre de notre dernière année de Master Professionnel en Science des Données à l’Institut Supérieur d’Informatique et de Mathématiques de Monastir (ISIMM). Il porte sur le développement de \textit{VitamiNurse}, une application innovante conçue pour aider les utilisateurs à faire des choix alimentaires éclairés et personnalisés. Basée sur l’intelligence artificielle, elle évalue les produits, suggère des alternatives plus saines et offre un accompagnement interactif en temps réel.

Développé à l’aide des méthodologies Scrumban et CRISP-DM, le projet VitamiNurse s’articule autour de plusieurs composants, dont un pipeline ETL, un système de recommandation hybride et un chatbot intelligent. Ensemble, ces éléments créent un outil dynamique et convivial permettant aux utilisateurs de faire des choix alimentaires plus sains et plus équilibrés.

\noindent\textbf{Mots-clés:} ETL, Systèmes de recommandation (SR), Apprentissage automatique, IA générative, chatbot, ANN et SentenceTransformers.