\newpage
\section*{General Conclusion}
\addcontentsline{toc}{section}{General Conclusion}
The VitamiNurse project represents a significant step forward in the integration of artificial intelligence into personalized nutrition and digital health. By combining a robust data pipeline, a hybrid recommendation system, and an intelligent conversational assistant, this work delivers a holistic solution that addresses key limitations observed in existing nutritional applications—namely, the lack of personalization, real-time adaptability, interactivity, and contextual awareness. 

Throughout this internship at ONRTECH, we successfully designed and implemented several core components that collectively transform VitamiNurse from a basic barcode scanner into an intelligent nutritional companion. The Nutri-pred-v1 model ensures data completeness by accurately predicting missing nutritional values, thereby enhancing the reliability of downstream analyses. The ETL pipeline enables continuous ingestion and enrichment of product data from diverse European sources, while standardization and multilingual handling ensure global applicability. 

The hybrid recommendation system—leveraging both collaborative filtering (via LightFM) and semantic content-based retrieval (via ChromaDB and Sentence Transformers)—delivers highly relevant, health-aware suggestions tailored to each user’s dietary restrictions, medical conditions, and evolving preferences. Benchmarking confirmed that the use of HNSW-based approximate nearest neighbor search provides an optimal balance between speed and accuracy, ensuring scalability for real-world deployment. 

Furthermore, the AI Assistant, built on a LangGraph-powered state machine and orchestrated through LangChain with GPT-4, introduces a dynamic, conversational interface capable of parallel intent processing, real-time profile updates, and personalized recipe generation. This assistant not only answers user queries but also learns continuously from interactions, offering increasingly precise and empathetic guidance over time. 

Together, these innovations position VitamiNurse as more than just an app—it becomes a proactive, adaptive, and trustworthy nutritional coach. The modular architecture ensures maintainability and extensibility, paving the way for future enhancements such as multi-modal input (e.g., image-based food recognition), integration with wearable devices, voice interaction, and reinforcement learning for long-term behavioral adaptation. 

In conclusion, this project demonstrates the transformative potential of AI in promoting healthier lifestyles through accessible, intelligent, and user-centric technology. It successfully bridges the gap between academic research in data science and practical industry application, fulfilling the objectives of the National Professional Master’s Degree in Computer Science while contributing meaningfully to the field of AI-driven digital health. 
 