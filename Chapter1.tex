\chapter{Project Context}
\section*{Introduction}
\addcontentsline{toc}{section}{Introduction}
\par This chapter introduces the foundational context for the development of \mbox{VitamiNurse}, a nutritional mobile application created during my internship at ONRTECH. 

Our goal is to develop a personalized recommendation system and an AI assistant for nutritional guidance. The system presents product details, assesses the scanned product with AI, and then provides a clear conclusion regarding the compatibility with the user’s nutritional profile.
The chapter also explains the basic theories behind the project , compares existing nutrition applications to show their limitations, and presents VitamiNurse as an innovative solution. Finally, we justify the use of the Scrumban methodology for team management and the CRISP-DM framework for technical development.

\section{Presentation of the host company}
\par ONRTECH\footnote{\url{https://onrtech.fr/}} is a French company, operating since 2020 and located in Villiers-sur-Marne, France, that focuses on software development, IT consulting, and artificial intelligence. It operates within the wider field of programming, advisory, and other digital services, delivering modern, innovative solutions tailored to evolving business needs.

\begin{figure}[H]
    \centering
    \includegraphics[width=0.3\textwidth]{images/ONRTECH_logo.jpeg}
    \caption{ONRTECH company logo}
    \label{fig:onrtech_logo}
\end{figure}
\par Since its creation, ONRTech has rapidly grown into a trusted provider of digital services, having collaborated with numerous clients throughout France and Europe.  More than just a tech provider,  it positions itself as a durable digital partner by focusing on customer satisfaction and offering customized solutions.


\section{Internship context and project objectives}
As part of our training at the Higher Institute of Computer Science and Mathematics in Monastir, we had the opportunity to complete our end-of-studies internship at ONRTECH.

 The goal of this internship is to develop a personalized recommendation system and an AI assistant for the nutritional mobile app Vitaminurse. The system presents product details, assesses the scanned product with AI, and then shows a clear conclusion of the analysis on whether the product is suitable for the user’s nutritional profile. If it is not suitable, the recommendation system suggests more adapted alternatives.

In addition, it engages users in friendly conversation and provides personalized nutritional advice. By understanding the preferences, health goals, and nutritional needs of each person, Vitaminurse provides personalized advice and recommendations that actually make sense for their lifestyle and health needs. 
\subsection{First version of VitamiNurse}
The initial version of VitamiNurse exists as a simple mobile application developed by ORTECH  to enable users scan product barcodes for viewing nutritional data. The application operates in a market which presents major obstacles to its success because of intense competition. The food tracking market already has two established apps Yuka and OpenFoodFacts which makes it difficult for VitamiNurse to get noticed. The app does not have special features that would help it stand out from other apps. The entire user interface operates with lower interactivity than top competitors, which results in decreased user involvement and reduced user retention. The study of current food monitoring applications together with their performance details will help identify user requirements for this domain.

\begin{center}
\begin{figure}[ht]
            \centering
            \includegraphics[scale=0.55]{images/VN_app_logo.png}
            \caption{VN app logo} 
            \label{fig:VN app logo}
        \end{figure}
\end{center}

\section{Problem Statement and Challenges}

The increasing focus on healthier lifestyles has driven demand for tools that support informed dietary choices. However, many individuals struggle to navigate the complexities of nutrition due to limitations in existing mobile applications. These tools often fail to provide personalized, reliable, and practical guidance, leaving users without the support they need to make balanced food choices. 
 
This section presents an analytical review of various applications in the field of food monitoring in order to reformulate a clear vision of an more advanced version of \mbox{VitamiN-urse}. The study is based on both the examination of relevant references and the comparative analysis of competing applications currently available in the market.


\newpage
\subsection{Analysis of Key Competitor Applications}
\subsection{Yuka}
With over 74,000 users, \footnote{\url{https://yuka.io/}} is a widely used nutritional mobile application, particularly in France and the United States. As shown in Figure \ref{fig:yuka_evaluation}, it scans food and cosmetic products to provide a quick quality rating: excellent, good, mediocre, or bad. Its intuitive interface and simple scoring system based on composition, additives, and organic status help users make quick purchasing decisions. 
Despite these advantages, Yuka also presents notable limitations. Its database is not always updated regularly. In addition, the application lacks personalization by overlooking individual dietary goals, allergies, and health conditions. As a result, Yuka serves more as a tool for initial screening than as a source of personalized nutritional guidance.
\begin{center}
\begin{figure}[H]
\includegraphics[scale=0.33]{images/yuka_evaluation.png}
\caption{The Yuka system for product evaluation}
\label{fig:yuka_evaluation}
\end{figure}
\end{center}

\subsection{ Open Food Facts}
Open Food Facts\footnote{\url{https://fr.openfoodfacts.org/}} is a nonprofit, collaborative, and open-data database of food products. It is the primary source of data for many other application, including Yuka, and provides detailed information. 

However, there are many limitations in the Open Food Facts app. Relying on user-generated content can lead to inaccuracies, duplicates, and incomplete data entries. The platform itself functions more as a powerful database than a user-friendly application. The lack of features for personal tracking, goal setting, or personalized recommendations limits its utility for consumers seeking a guided health journey.

% OFF 00-01-02-03
\begin{figure}[H]
    \centering
    \includegraphics[width=0.22\textwidth]{images/OFF-00.png}
    \includegraphics[width=0.22\textwidth]{images/OFF-01.png}
    \includegraphics[width=0.22\textwidth]{images/OFF-02.png}
    \includegraphics[width=0.22\textwidth]{images/OFF-03.png}
    \caption{Lack of personalization and user-friendly features in the OFF app}
    \label{fig:natilait_screenshots}
\end{figure}

\subsection{QuelProduit}
QuelProduit\footnote{\url{https://www.quechoisir.org/application-mobile-quelproduit-n84731/}} is a product analysis app developed by the French non-profit consumer organization \textbf{UFC-Que Choisir}, to help users make safer choices by scanning barcodes to evaluate food, cosmetic, and household items.
 The application focuses on consumer safety by highlighting harmful substances, but does not offer balanced nutritional guidance or personalized dietary support. Users should view this tool as a hazard detection system rather than a nutritional guide or assistant.
 
\subsection{Key Challenges and gaps in Current Applications}
There are a number of significant deficiencies of current nutritional apps. Firstly, existing nutrition applications frequently fall to deliver a comprehensive user experience. Many provide generic recommendations that do not account for individual factors such as allergies and medical conditions. This lack of personalization can make it difficult for users to find relevant and actionable advice. Furthermore, the widespread use of simplified scoring systems further reduces complex nutritional data to a single value, which can mislead users and prevent informed decision-making.  

Secondly, there is a lack of interactivity, missing real-time guidance via chatbots to answer questions and provide tailored recommendations and advice. The lack of conversational guidance restricts the ability of apps to provide dynamic and context-aware support. 

 Lastly, data reliability issues still persist, raising concerns about the precision and reproducibility of nutritional data in existing databases. The negative user feedback on Google Play points to outdated Yuka database, as illustrated in Figure \ref{fig:yuka_feedback}. This results in unreliable nutritional analysis and inconsistent product ratings.  

\begin{center}
\begin{figure}[ht]
\includegraphics[scale=0.45]{images/yuka_negative_review.png}
\caption{Negative user reviews of Yuka on Google Play}
\label{fig:yuka_feedback}
\end{figure}
\end{center}

  
\subsection{Proposed solution}
To address these gaps, we propose a comprehensive enhancement of VitamiNurse centered on four interconnected technical pillars. First, a robust ETL pipeline will ensure database reliability through automated web scraping and data integration. As a result of this pipeline, it is ensured that product information is regularly updated and systematically verified against trusted sources.
Second, a hybrid recommendation system designed to solve personalization problems. It combines collaborative filtering using the LightFM library with semantic similarity search based on a vector database and HNSW graphs. 
 Third, an AI-powered nutrition analysis system will audit the scanned product and detect those deemed unhealthy or unsuitable for the user’s specific dietary profile and health goals.
 Finally, an interactive AI chatbot will bridge the interactivity gap. Not only will it engage users in dialogue to answer nutritional questions, but it will also understand their needs in depth and provide personalized guidance. This integrated architecture is designed to transform VitamiNurse from a simple product scanner into an intelligent and adaptive nutritional assistant.

\begin{center}
\begin{figure}[H]
            \centering
            \includegraphics[scale=0.11]{images/VN_AI_solution.jpg}
            \caption{VitamiNuse login page} 
            \label{fig:The AI powered app : VitamiNurse}
        \end{figure}
\end{center}


\section{Project Methodology}

The first step to ensure the smooth running and success of any project is choosing the right methodology. Today,Scrum, XP, and Kanban and  Scrumban are the most popular agile approaches in web and mobile development.
To choose the best approach, we decided to compare and select the one that best fits our context and objectives.

\subsection{Comparative Study of existing methodologies}
\subsubsection{a) Scrum Method} 
Scrum is an agile methodology that promotes collaboration, adaptability, and the iterative delivery of high-quality results. This method encourages agile teams to meet regularly to review project progress and adjust their direction as needed. This dynamic approach balances investment and the final product, ensuring greater customer satisfaction and optimal value delivery \cite{scrum2025}.

The main elements of Scrum are:
\begin{enumerate}
    \item \texttt{Sprint:} Is a working cycle with fixed duration, of one to four weeks. In this case, it is the Scrum Master to ensure that this period is not exceeded \cite{schwaber2004agile}

    \item \texttt{Scrum Master:} Their main role is to facilitate the application of Scrum principles and resolve blockages.

    \item \texttt{Scrum Team:} A multidisciplinary, self-organizing team that manages the planning and delivery of features.


    \item \texttt{Product Owner:} He validates the functionalities, prioritizes the requirements and user expectations.

    
\end{enumerate}
\subsubsection{b) Kanban Method}
Kanban, originating from Japanese industry, is a visual workflow management tool. It allows you to visualize and limit work in progress in software development and IT work . The main elements of Kanban are as follows:

Kanban, originating from Japanese industry, is a visual workflow management tool. It allows you to visualize and limit work in progress in software development and IT work\cite{anderson2010kanban}. The main elements of Kanban are as follows:A \texttt{Kanban board} visualizes workflow steps ("To do", "In progress", "Done"), \texttt{Kanban cards} represent tasks with their details (description, deadlines), and \texttt{WIP} (Work In Progress) reduces overloads and quickly identifies bottlenecks.

\subsubsection{c) XP Method (eXtreme Programming)}

XP takes agile principles to the extreme with rapid iterations and enhanced customer collaboration with minimal documentation. It is characterized by frequent code reviews and a creative and collaborative working environment through pair programming\cite{beck2000extreme}. 


By performing frequent code reviews and unit tests to make changes quickly, it seeks the best quality and customer satisfaction that remains in constant contact throughout the project. This approach is based on 5 values: \texttt{Simplicity}, which is favored by the search for easier solutions to achieve objectives; \texttt{communication} in direct communication between team members on the one hand and the client on the other; \texttt{feedback} through the constant exchange of feedback between the project team and key customers; \texttt{courage} required to undertake changes such as exploring new paths; and \texttt{respect} for each team member and their work.


\subsubsection{d) Scrumban Method}

\begin{center}
\begin{figure}[H]
            \centering
            \includegraphics[scale=0.44]{images/scrumban.jpg}
            \caption{Scrumban Process }
            \label{fig:Scrumban_Process}
\end{figure}
\end{center}

Scrumban is a hybrid methodology that combines the flexibility of Kanban board with the rigor of Scrum sprints. Scrum roles are incorporated while enabling real-time modifications and the best possible workflow visualization. The team does, in fact, follow certain rituals, use sprints, and assign roles to the Scrum Master and Product Owner. Nonetheless, the procedure is less exacting than Scrum, enabling teams to maintain an ideal workflow and make real-time priority adjustments.
   


\subsection{Chosen Team Management Methodology: Scrumban}
Our selection of the Scrumban methodology is based on an analysis of the limitations of pure approaches (Scrum, XP, Kanban) in the practical plan in the host company as well as the specific needs of our project. Generally, we can summarize the identified problems as follows:
\begin{itemize}
    \item \textbf{Limited numbers:}
    The XP (eXtreme Programming) approach requires development pairs (pair programming), which is difficult to apply with the available human resources.
        
    \item \textbf{Accelerated Scrum Pace:}
    Short sprints (2–4 weeks) can generate excessive pressure on the development team, risking compromising quality, especially for an end-of-study internship.    
        
    \item \textbf{Flexibility needs:}
    The rigidity of Scrum rituals (mandatory sprint planning) is unsuitable for the rapid evolution of artificial intelligence technologies and the frequent changes imposed by stakeholders.
    \end{itemize}
    
\begin{center}
\begin{figure}[H]
            \centering
            \includegraphics[scale=0.38]{images/kanbanBoard.png}
            \caption{Our Scrumban Board Implementation}
            \label{fig:Tableau_Scrumban}
\end{figure}
\end{center}

The first step in the Scrumban process is to set up the Scrumban board, which is an enhanced version of the Kanban board. It contains all the features, including the product backlog, sprint backlog, and workflow stages (not started, in progress, and under review).




\newpage
\subsection{Chosen technical methodology}

We adopted the CRISP-DM as a technical methodology for its structured and iterative approach to solving complex data science problems. It represents a comprehensive six-phase process includes problem understanding, data exploration, data preparation, modeling, evaluation, and deployment.
\begin{center}
\begin{figure}[H]
            \centering
            \includegraphics[scale=0.44]{images/CRISP.png}
            \caption{CRISP-DM} 
            \label{fig:CRIS-DM}
        \end{figure}
\end{center}




\section*{Conclusion}
\addcontentsline{toc}{section}{Conclusion}
In this first chapter, we discussed the general context of the Vitaminurse project. We also discussed the limitations of existing applications on the market compared to our application, which leverages more artificial intelligence to improve user nutritional support.

We also defined the methodologies adopted for the project's development. For team organization, the Scrumban method was chosen for its flexibility and clarity. CRISP-DM was chosen as a technical work methodology, as it provides a structured approach for data science projects.

This chapter provides a solid foundation for the rest of the work before moving on to the next chapter, dedicated to the development of the data pipeline.

